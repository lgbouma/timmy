%%%%%%%%%%%%%%%%%%%%%%%%%%%%%%%%%%%%%%%%%%%%%%%%%%%%%%%%%%%%%%%%%%%%%%%%%%%%%%%

\documentclass[12pt,twocolumn,tighten]{aastex62}
%\documentclass[12pt,twocolumn,tighten,trackchanges]{aastex62}
\usepackage{amsmath,amstext,amssymb}
\usepackage[T1]{fontenc}
\usepackage{apjfonts}
\usepackage[figure,figure*]{hypcap}
\usepackage{graphics,graphicx}
\usepackage{hyperref}
\usepackage{natbib}
\usepackage[caption=false]{subfig} % for subfloat
\usepackage{enumitem} % for specific spacing of enumerate
\usepackage{epigraph}

\renewcommand*{\sectionautorefname}{Section} %for \autoref
\renewcommand*{\subsectionautorefname}{Section} %for \autoref

\newcommand{\tn}{TOI~837} % target star name
\newcommand{\pn}{TOI~837.01} % planet name
\newcommand{\cn}{IC~2602} % cluster name


%% Reintroduced the \received and \accepted commands from AASTeX v5.2.
%% Add "Submitted to " argument.
\received{\today}
\revised{---}
\accepted{---}
\submitjournal{AAS journals.}
\shorttitle{TOI~837 in IC~2602}

\begin{document}

\defcitealias{bouma_wasp4b_2019}{B19}

\title{Cluster Difference Imaging Photometric Survey (CDIPS) II: A Jupiter-Sized Planet in IC~2602}

\correspondingauthor{L.\,G.\,Bouma}
\email{luke@astro.princeton.edu}

%
% key contributions
%
\author[0000-0002-0514-5538]{L. G. Bouma}
\affiliation{Department of Astrophysical Sciences, Princeton University, 4 Ivy Lane, Princeton, NJ 08540, USA}

\author[0000-0001-8732-6166]{J. D. Hartman}
\affiliation{Department of Astrophysical Sciences, Princeton University, 4 Ivy Lane, Princeton, NJ 08540, USA}

%OK
\author[0000-0002-9158-7315]{R. Brahm} %rafael.brahm@uai.cl coordination, WINE board
\affiliation{Facultad de Ingenier\'{i}a y Ciencias, Universidad Adolfo Ib\'a\~nez, Av.\ Diagonal las Torres 2640, Pe\~nalol\'en, Santiago, Chile}
\affiliation{Millennium Institute for Astrophysics, Chile}

%OK
\author[0000-0002-5674-2404]{P. Evans}
\affiliation{El Sauce Observatory, Coquimbo Province, Chile}

%OK
\author[0000-0001-6588-9574]{K. A. Collins} % karenacollins@outlook.com
\affiliation{Center for Astrophysics \textbar \ Harvard \& Smithsonian, 60 Garden St, Cambridge, MA 02138, USA}

%OK
\author[0000-0002-4891-3517]{G. Zhou}
\affiliation{Center for Astrophysics \textbar \ Harvard \& Smithsonian, 60 Garden St, Cambridge, MA 02138, USA}

%OK
\author[0000-0001-8128-3126]{P. Sarkis} %Paula sarkis@mpia.de PI of FEROS time
\affiliation{Max-Planck-Institut f\"{u}r Astronomie, K\"onigstuhl 17, Heidelberg 69117, Germany }

%OK
\author[0000-0002-8964-8377]{S. N. Quinn} % squinn@cfa.harvard.edu
\affiliation{Center for Astrophysics \textbar \ Harvard \& Smithsonian, 60 Garden St, Cambridge, MA 02138, USA}

%OK
\author{J. de Leon}
\affiliation{Department of Astronomy, University of Tokyo, 7-3-1
Hongo, Bunkyo-ky, Tokyo 113-0033, Japan}

%OK
\author[0000-0002-4881-3620]{J. Livingston}
\affiliation{Department of Astronomy, University of Tokyo, 7-3-1
Hongo, Bunkyo-ky, Tokyo 113-0033, Japan}

%OK
\author[0000-0003-2989-7774]{C. Bergmann}
\affiliation{Exoplanetary Science at UNSW, School of Physics, UNSW
Sydney, NSW 2052, Australia}
\affiliation{Deutsches Zentrum f\"ur Luft- und Raumfahrt, M\"unchener
Str. 20, 82234 Wessling, Germany}

%OK
\author[0000-0002-3481-9052]{K. G. Stassun}
\affiliation{Vanderbilt University, Department of Physics \& Astronomy, 6301 Stevenson Center Lane, Nashville, TN 37235, USA}
\affiliation{Fisk University, Department of Physics, 1000 17th Avenue N., Nashville, TN 37208, USA}
%

%
% Princeton team
%
%OK
\author[0000-0002-0628-0088]{W. Bhatti}
\affiliation{Department of Astrophysical Sciences, Princeton University, 4 Ivy Lane, Princeton, NJ 08540, USA}
%
\author[0000-0002-4265-047X]{J. N. Winn}
\affiliation{Department of Astrophysical Sciences, Princeton University, 4 Ivy Lane, Princeton, NJ 08540, USA}
%
\author[0000-0001-7204-6727]{G. \'A. Bakos}
\affiliation{Department of Astrophysical Sciences, Princeton University, 4 Ivy Lane, Princeton, NJ 08540, USA}

%
% SG1 contributors
%

% begin (ASTEP team)
%OK
\author{L. Abe} % Lyu
\affiliation{Universit\'e C\^{o}te d'Azur, Observatoire de la C\^ote d'Azur, CNRS, Laboratoire Lagrange, Bd de l'Observatoire, CS 34229, 06304 Nice cedex 4, France}

%%PENDING->OMITTED
%\author{A. Agabi} % Abdelkrim 
%\affiliation{Universit\'e C\^{o}te d'Azur, Observatoire de la C\^ote d'Azur, CNRS, Laboratoire Lagrange, Bd de l'Observatoire, CS 34229, 06304 Nice cedex 4, France}

%OK
\author[0000-0001-7866-8738]{N. Crouzet} % Nicolas
\affiliation{European Space Agency, European Space Research and Technology Centre (ESA/ESTEC), Keplerlaan 1, 2201 AZ Noordwijk, The Netherlands}

%OK
\author[0000-0002-3937-630X]{G. Dransfield} % Georgina
\affiliation{School of Physics \& Astronomy, University of Birmingham, Edgbaston, Birmingham B15 2TT, United Kingdom}

\author[0000-0002-7188-8428]{T. Guillot} % Tristan 
\affiliation{Universit\'e C\^{o}te d'Azur, Observatoire de la C\^ote d'Azur, CNRS, Laboratoire Lagrange, Bd de l'Observatoire, CS 34229, 06304 Nice cedex 4, France}

%OK
\author{W. Marie-Sainte} % Wenceslas
\affiliation{Institut Paul \'{E}mile Victor, Concordia Station, Antarctica}

%OK
\author{D. M\'ekarnia} % Djamel
\affiliation{Universit\'e C\^{o}te d'Azur, Observatoire de la C\^ote d'Azur, CNRS, Laboratoire Lagrange, Bd de l'Observatoire, CS 34229, 06304 Nice cedex 4, France}

%OK
\author[0000-0002-5510-8751]{A. H.M.J. Triaud} % Amaury
\affiliation{School of Physics \& Astronomy, University of Birmingham, Edgbaston, Birmingham B15 2TT, United Kingdom}
% end (ASTEP team)

%
% SG2 contributors
%


%OK
\author[0000-0002-7595-0970]{C.~G.~Tinney}
\affiliation{Exoplanetary Science at UNSW, School of Physics, UNSW Sydney, NSW 2052, Australia}


%
% WINE MPIA team
%

%OK
\author{T. Henning} %Thomas henning@mpia.de FEROS, WINE board
\affiliation{Max-Planck-Institut f\"{u}r Astronomie, K\"onigstuhl 17, Heidelberg 69117, Germany }

%OK
\author[0000-0001-9513-1449]{N. Espinoza} %N\'estor nespinoza@stsci.edu WINE board
\affiliation{Space Telescope Science Institute, 3700 San Martin Drive, Baltimore, MD 21218, USA}

%OK
\author[0000-0002-5389-3944]{A. Jord\'an} %andres.jordan@gmail.com WINE board
\affiliation{Facultad de Ingenier\'{i}a y Ciencias, Universidad Adolfo Ib\'a\~nez, Av.\ Diagonal las Torres 2640, Pe\~nalol\'en, Santiago, Chile}
\affiliation{Millennium Institute for Astrophysics, Chile}

%OK
\author{M. Barbieri} %Maruo mauro.barbieri@uda.cl FEROS observations
\affiliation{INCT, Universidad de Atacama, calle Copayapu 485, Copiap\'o, Atacama, Chile}

%OK
\author{S. Nandakumar} %sangeetha.nandakumar@postgrados.uda.cl  FEROS observations
\affiliation{INCT, Universidad de Atacama, calle Copayapu 485, Copiap\'o, Atacama, Chile}

%OK
\author{T. Trifonov} %Trifon trifonov@mpia.de FEROS observations
\affiliation{Max-Planck-Institut f\"{u}r Astronomie, K\"onigstuhl 17, Heidelberg 69117, Germany }

%OK
\author[0000-0002-1896-2377]{J.~I.~Vines} %Jose jose.vines.l@gmail.com FEROS observations
\affiliation{Departamento de Astronom\'ia, Universidad de Chile, Camino El Observatorio 1515, Las Condes, Santiago, Chile}

%OK
\author{M. Vuckovic} %Maja maja.vuckovic@uv.cl FEROS observations
\affiliation{Instituto de F\'isica y Astronom\'ia, Universidad de Vapara\'iso, Casilla 5030, Valpara\'iso, Chile}

%
% SG4 contributors
%
%OK
\author[0000-0002-0619-7639]{C.~Ziegler} % carl.ziegler@dunlap.utoronto.ca
\affiliation{Dunlap Institute for Astronomy and Astrophysics, University of Toronto, 50 St. George Street, Toronto, Ontario M5S 3H4, Canada}

%%OMITTED
%\author{C.~Brice\~{n}o} % cbriceno@ctio.noao.edu
%\affiliation{Cerro Tololo Inter-American Observatory, Casilla 603, La Serena, Chile}

%OK
\author{N.~Law} % nlaw@unc.edu
\affiliation{Department of Physics and Astronomy, The University of North Carolina at Chapel Hill, Chapel Hill, NC 27599-3255, USA}

%OK
\author[0000-0003-3654-1602]{A.~W.~Mann} % awmann@unc.edu
\affiliation{Department of Physics and Astronomy, The University of North Carolina at Chapel Hill, Chapel Hill, NC 27599-3255, USA}


% 
%-------------------------------------
% TESS Mission Architects:
% These authors should be listed in this order
% see https://spacebook.mit.edu/pages/viewpage.action?pageId=24543276
%-------------------------------------
%
%OK
\author{G. R. Ricker} % grr@space.mit.edu
\affiliation{Department of Physics and Kavli Institute for Astrophysics and Space Research, Massachusetts Institute of Technology, Cambridge, MA 02139, USA}
%
%OK
\author[0000-0001-6763-6562]{R. Vanderspek} % roland@space.mit.edu
\affiliation{Department of Physics and Kavli Institute for Astrophysics and Space Research, Massachusetts Institute of Technology, Cambridge, MA 02139, USA}
%
% %OMITTED
% \author[0000-0001-9911-7388]{D. W.~Latham} % dlatham@cfa.harvard.edu
% \affiliation{Center for Astrophysics \textbar \ Harvard \& Smithsonian, 60 Garden St, Cambridge, MA 02138, USA}
%
%OK
\author{S. Seager} % seager@mit.edu
\affiliation{Department of Earth, Atmospheric, and Planetary Sciences, Massachusetts Institute of Technology, Cambridge, MA 02139, USA}
%
%OK
\author[0000-0002-4715-9460]{J. M.~Jenkins} % jon.jenkins@nasa.gov
\affiliation{NASA Ames Research Center, Moffett Field, CA 94035, USA}

%
%-------------------------------------
% 3 representatives of each of SPOC, POC, TSO, for a total of 9. 
%These 9 authors should be listed in alphabetical order
%-------------------------------------


%OK
\author[0000-0002-7754-9486]{C.~J.~Burke} % cjburke@mit.edu
\affiliation{Department of Physics and Kavli Institute for Astrophysics and Space Research, Massachusetts Institute of Technology, Cambridge, MA 02139, USA}

%OK
\author[0000-0003-2313-467X]{D.~Dragomir}
\affiliation{Department of Physics and Astronomy, University of New Mexico, Albuquerque, NM, USA}
	
%%PENDING->OMITTED
%\author[0000-0003-0918-7484]{C.~X.~Huang}
%\affiliation{Department of Physics and Kavli Institute for Astrophysics and Space Research, Massachusetts Institute of Technology, Cambridge, MA 02139, USA}

%%PENDING->OMITTED
%\author{R.~C.~Kidwell, Jr.} % rkidwell@stsci.edu
%\affiliation{Space Telescope Science Institute, 3700 San Martin Drive, Baltimore MD 21218 }

%OK
\author[0000-0001-8172-0453]{A.~M.~Levine} % aml@space.mit.edu
\affiliation{Department of Physics and Kavli Institute for Astrophysics and Space Research, Massachusetts Institute of Technology, Cambridge, MA 02139, USA}

%OK
\author{E.~V.~Quintana} % elisa.quintana@nasa.gov
\affiliation{NASA Goddard Space Flight Center, 8800 Greenbelt Road, Greenbelt, MD 20771, USA}

%OK
\author[0000-0001-8812-0565]{J.~E.~Rodriguez}
\affiliation{Center for Astrophysics \textbar \ Harvard \& Smithsonian, 60 Garden St, Cambridge, MA 02138, USA}

%OK
\author[0000-0002-6148-7903]{J. C. Smith} % jeffrey.c.smith-1@nasa.gov
\affiliation{NASA Ames Research Center, Moffett Field, CA 94035, USA}
\affiliation{SETI Institute, Mountain View, CA 94043, USA}

%OK
\author[0000-0002-5402-9613]{B. Wohler} % bill.wohler@nasa.gov
\affiliation{NASA Ames Research Center, Moffett Field, CA 94035, USA}
\affiliation{SETI Institute, Mountain View, CA 94043, USA}


\begin{abstract}
We report the discovery of the transiting planet TOI 837.01,
a warm Jupiter-sized planet ($R_{\rm p} = 10.XX\,R_\oplus$, $P =
8.YY\,{\rm d}$), with mass less than X.XX $M_{\rm Jup}$.
TOI$\,$837.01 orbits a V=10.6, T=9.9 G-dwarf in IC~2602, also known as the
``Southern Pleiades'', and is therefore 50 million years old.
We characterize the system using data from the NASA {\it Transiting
Exoplanet Survey Satellite} (TESS), the ESA Gaia mission,
ground-based photometry, and spectroscopy from CTIO1.5/CHIRON,
XXX/FEROS, and AAT/Veloce.
TOI 837.01 joins the growing ranks of adolescent exoplanets orbiting
bright host stars, and is amenable for studies of planetary
atmospheric and orbital evolution.
\end{abstract}

\keywords{
	Exoplanets (XXX),
	Exoplanet evolution (491),
	Stellar ages (1581),
	Young star clusters (XXX)
}

%%%%%%%%%%%%%%%%%%%%%%%%%%%%%%%%%%%%%%%%%%%%%%%%%%%%%%%%%%%%%%%%%%%%%%%%%%%%%%%


\section{Introduction}

Recently, young planets have been hype.

Section~\ref{sec:observations} 
Section~\ref{sec:star}
Section~\ref{sec:planet}
Section~\ref{sec:discussion}
Section~\ref{sec:conclusions}.



\section{The Data}
\label{sec:observations}

\begin{figure*}[t!]
	\begin{center}
		\leavevmode
		\includegraphics[width=1\textwidth]{f1.pdf}
	\end{center}
	\vspace{-0.7cm}
	\caption{
    {\bf TESS lightcurve of \tn\ (Sectors 10 and 11).}
		{\it Top}: \texttt{PDCSAP} mean-subtracted relative flux at 2-minute
		sampling. Spot-induced stellar variability is the dominant signal.  Dashed
		lines show the five transits observed by TESS.
    {\it Bottom}: Zoomed windows of individual transits.
		\label{fig:rawzoom}
	}
\end{figure*}

\subsection{TESS Photometry}

\tn\ (TIC 460205581, GAIA DR2 5251470948229949568) was observed by the
TESS spacecraft from X to Y in two-minute cadence mode.

Figure~\ref{fig:rawzoom}.

\subsection{Ground-based Time-Series Photometry}

\subsection{Imaging}

\subsection{Spectroscopy}

\subsubsection{CHIRON}
9 spectra with CTIO1.5/CHIRON, from January 28 through March 14, 2020.
6 were usable.

\subsubsection{FEROS}
N spectra with XXX/FEROS

\subsubsection{Veloce}
M spectra with AAT/Veloce.

\subsection{Astrometry}


\section{The Star}
\label{sec:star}

\subsection{The Cluster}
\label{subsec:cluster}

% NOTE: might want a table of all ages

IC~2602 canonically has an age of $30\pm20\,{\rm Myr}$ (CITE: Van
Leeuwen 2009).
Or logage between 7.533 and 7.563	\citep{bossini_age_2019}.

Or Li age of 40-50Myr (David+19, comparison with other Li stars).

Reported mean metallicity values ${\rm [Fe/H]}$ for the cluster range between
slightly super-solar ($0.04\pm0.01$, \citealt{baratella_gaia-eso_2020}) and
slightly sub-solar ($-0.02 \pm 0.02$, \citealt{netopil_metallicity_2016}).

IC~2602 is supervirial, in the sense that the observed stellar velocity
dispersion is larger than the value expected if it were in virial equilibrium
by about a factor of two \citep{bravi_gaia-eso_2018}.


\subsection{Cluster Membership}
\label{subsec:member}

The Gaia kinematics are good; gamma velocity is correct; Li (FEROS +
Veloce) is strong; the rotation period and vsini agree with
sub-Pleiades age expected for an IC 2602 member.

\subsubsection{Kinematics}

Gaia

\subsubsection{Rotation}

TESS photometric rotation period = 3.5days ish.

vsini = 17.48 $\pm$ 0.15 (CHIRON).

The two agree.

For comparison, gyrochrones would predict X.XX.

This implies sub-Pleiades age expected for an IC 2602 member.

\subsubsection{Lithium}

Rafael's co-added FEROS spectra gave a Li doublet equivalent width of
154 milliangstrom (see attached).

As a reminder, Teff=6100K (TICv8). 
Or Teff = 5946 $\pm$ 39 (CHIRON).
Rafael, if you have a better
determination of Teff, it would be helpful!  SpType is early-G,
late-F from your quicklook spectral typing earlier.

Relevant colors are V-K = 1.7,  V-I ~= V-T = 0.7

Summarizing relevant lithium facts:

Lithium depletion for a star of TOI-837's spectral type happens
over timescales >100 Myr (Soderblom+2013). This is because the
convective envelopes are very shallow, so they don't cycle material
down to the >3e6 K core until much later.

Nonetheless, comparison of early-G field stars to e.g., 600 Myr old
Hyads shows that the depletion does happen over gigayear timescales
(Berger+2018, fig 7).

Comparing to the Li EWs measured by Berger+2018 (their Fig 5),
TOI-837 is consistent with being younger than almost all CKS
planet-hosting field stars.

The TOI-837 Li EW measured here (150 milliangstrom) is consistent
with what's seen for stars of similar colors in <100 Myr old moving
groups (Elliott+2016, Fig 11 -- attached).  This is sensible, because
IC 2602 is supposed to be 50 Myr old.


\subsubsection{Literature Membership}
The membership of \tn\ in \cn\ was noted by \citet{oh_comoving_2017},
in what they dubbed ``Group 5''.
Analyzing the clusters of \citet{oh_comoving_2017},
\citet{bochanski_fundamental_2018} fitted the Gaia, 2MASS, and
WISE photometry with MIST isochrones, and reported stellar masses,
radii, metallicities, ages, distances, and extinction for 9754 stars,
including \tn.

\tn\ was also listed as a member of ``Theia 92'' by
\citet{kounkel_untangling_2019}.
These authors identified 999 candidate cluster members, and
reported a cluster isochrone age $\log t = 7.55$ with uncertainty
$\approx 0.15\,{\rm dex}$.


\subsection{Stellar Parameters}

\begin{figure}[t!]
	\begin{center}
		\leavevmode
		\includegraphics[width=0.48\textwidth]{f6.png}
	\end{center}
	\vspace{-0.7cm}
	\caption{
    {\bf Spectral energy distribution of \tn\ from archival
    photometry.}
    {\bf Keivan: please describe!}
    \label{fig:sed}
	}
\end{figure}

\subsubsection{Physical}
 (Lstar, Rstar, Teff, Age and Mstar)

Figure~\ref{fig:sed}.

\paragraph{Astrometric information}
The RUWE (CITE, CITE) for \tn\ is 1.02, indicative that there are no
obviously present astometric companions.


\subsubsection{Rotation}
Stellar vsini
Stellar rotation

\subsection{RVs}
I don't strongly expect the Veloce RVs or the FEROS ones to lead to a
mass measurement. The reason is that with vsini=17km/s, and 2\%
rotation amplitude signal, we expect an RV RMS of 300m/s at the
Prot=3.5 day rotation period (vsini*rotation amplitude). This is
probably larger than the expected planet signal (100m/s, 8 days).



\section{The Planet}
\label{sec:planet}

\begin{figure}[t]
	\begin{center}
		\leavevmode
		\includegraphics[width=0.45\textwidth]{f3.pdf}
	\end{center}
	\vspace{-0.7cm}
	\caption{ {\bf Scene used for blend analysis.}
    {\it Top:} Mean TESS image of \tn\ over Sector~10, with a
    logarithmic grayscale. The yellow star is the position of \tn.
    Orange circles are neighboring stars with $T<16$, scaled such that
    brighter stars are larger. The \texttt{X} and \texttt{/} hatches
    show the apertures used to measure the background and target star
    flux, respectively. Dashed lines of constant declination and right
    ascension are shown.  {\it Bottom:} Digitized Sky Survey $R$-band
    image of the same field, with a linear grayscale. The circles show
    apertures of radii 1, 1.5, and 2.25 pixels used in our blend
    analysis (Section~\ref{subsec:validation}).  Two stars of interest
    are ``Star A'' and ``Star B'', which were eventually excluded as
    being possible sources of the transits.
		\label{fig:scene}
	}
\end{figure}

\begin{figure}[t]
	\begin{center}
		\leavevmode
		\includegraphics[width=0.45\textwidth]{f4.pdf}
	\end{center}
	\vspace{-0.7cm}
	\caption{ 
		{\bf SOAR HRCam contrast limits derived from point-source
			injection-recovery experiments.} 
      Star A ($\Delta T=4.7$, $\rho=2.3$'' west) is detected in the
      autocorrelation function, in addition to being a resolved Gaia
      source.
      It is co-moving with \tn, and
      its parallax and on-sky position imply that
      it is physically separated from \tn\ by $6.6\pm 0.1\,{\rm pc}$.
		\label{fig:soar}
	}
\end{figure}



\subsection{Transit Fitting}
Simultaneous dip plus rotation period (GP) fit. Use celerite plus PyMC3/exoplanet.
Mention FPP?


\subsection{Validation of \pn}

Exclude possibilities:
"The transits are blended from a background, unassociated eclipsing binary
system or transiting exoplanet: We take the same approach outlined in
Vanderburg et al. (2019). In short, if the transits are a blend from the
background system, the true radius ratio is constrained by the ratio of the
ingress time (T12) to the..."


\subsubsection{Visual Binarity}
Figure~\ref{fig:scene} shows the scene.
In the upper panels, the pixels used to measure the background level
in the SPOC lightcurve are indicated with `\texttt{X}' hatching, and
the pixels used in the final lightcurve aperture are shown with
`\texttt{/}' hatching.

Stars brighter than $T=16$, as queried from the Gaia DR2 source
catalog, are shown with orange circles.
The relevant stars for a blend analysis are as follows.
\begin{itemize}
  \item \tn\ = TIC 460205581 (T=9.9). The target star.
  \item Star A = TIC 847769574 (T=14.6). $2.3$'' west. Proper motions
    and parallax imply it is comoving with \tn, though with a physical
    separation of $6.6\pm 0.1\,{\rm pc}$, it may not be bound.
  \item Star B = TIC 460205587 (T=13.1). $5.4$'' north.  This is a
    giant background star.
\end{itemize}
An additional source, TIC 847769581, is $4.9$'' from the target, but
too faint (T=18.8) to be the source of the observed transit signal.

The Gaia DR2 data for Star A seems somewhat poorly behaved.  While
Star A has $G=15.1$, and $Bp=14.9$, no $Rp$ magnitude is reported.
Correspondingly, no RUWE value is available (CITE).  The astrometric
reduced $\chi^2$ ($\chi^2 / (N-5)$, for $N$ the number of good
astrometric observations) seems rather poor, at $8.6$.  We suspect
that the photometric failure to produce an $Rp$ magnitude, as well as
the poor astrometric fit, are likely due to blending with \tn.

At the angular resolution of the TESS data, if either Star A or Star B
were eclipsing binaries, they could be the sources of the transit
signal.  Coincidence stellar sources below the resolution limit of
Gaia are also a concern (CITE). To rule out these possibilities, we
took a number of approaches.


\subsubsection{High-resolution imaging}

We imaged the system using SOAR-HRCam.
{\bf Carl: please describe}.
Figure~\ref{fig:soar}.
\citet{ziegler_soar_2020}


\subsubsection{TESS analysis}

We examined the CDIPS FFI lightcurves of the target, which are
available on MAST \citep{bouma_cluster_2019}. Three lightcurves are
available, based on photometric apertures with a radius of 1, 1.5, or
2.5 pixels.  In the raw difference-image light-curves, as well as the
PCA-detrended light-curves, dips of depth $\approx$0.35\% are visible,
and their properties do not significantly vary with aperture size.
Within $\approx20$'', the dips are consistent with originating from
the target star.

\subsubsection{Seeing-limited time-series photometry}

%FIXME:
{\bf left off here}
Co-author P{.} Evans obtained a total transit that showed that, within
1.5'', the transit comes from \tn.  We can be even more certain with a
pixel-level analysis. This 2.1 arcsec neighbor is also a mid to late M
dwarf, and bound given the Gaia parallax + proper motion. The transit
duration is rather long for the signal to come from the M dwarf,
rather than the G dwarf.  I think the state of photometric followup
could be improved (there is only one good total transit), but I think
over this month and next we can pull another few partials.

A total eclipse of Ziegler's faint companion would produce a 1.4\%
dip. A 30\% eclipse, more plausible for a NEB scenario, would produce
dips at exactly the right depth (0.4\%).

\subsubsection{Summary}

Possible false-positive scenarios are as follows.

\begin{itemize}
  \item {\it Neighboring blends}.

  \item {\it Hierarchical blends}.
    Limited by i) FEROS + Veloce RVs, and ii) the star being nicely on the
    cluster HR diagram. Any extra hidden close-in binary stars would
    need to be very low-mass M dwarfs, and then the transit duration
    again helps.

  \item {\it Background blends}.
  Archival imaging (POSS) available in the south?
\end{itemize}


\subsection{Additional Companions}

None of the extra dips in the PDCSAP light-curve (e.g., ``up-down''
spikes at BTJD 1572 and 1601) seem likely to be planetary
We checked that i) they were not present in the SAPFLUX light-curves,
and ii) that they were not present in the CDIPS light-curves (either
raw, or PCA-detrended).

To make this quantitative, we did injection-recovery.



\section{Discussion}
\label{sec:discussion}

In context,...


\section{Conclusions}
\label{sec:conclusions}




%%%%%%%%%%%%%%%%%%%%%%%%%%%%%%%%%%%%%%%%%%%%%%%%%%%%%%%%%%%%%%%%%%%%%%%%%%%%%%%

\acknowledgements
%
%This paper includes data collected by the TESS mission, which are
%publicly available from the Mikulski Archive for Space Telescopes
%(MAST).
%
%Funding for the TESS mission is provided by NASA's Science Mission
%directorate.
%
The authors thank...
%
We also thank the Heising-Simons Foundation for
their generous support of this work.
%
The Digitized Sky Survey was produced at the Space Telescope Science
Institute under U.S. Government grant NAG W-2166.
Figure~\ref{fig:scene} is based on photographic data obtained using
the Oschin Schmidt Telescope on Palomar Mountain.
%
% %
% Based on observations obtained at the Gemini Observatory, which is
% operated by the Association of Universities for Research in Astronomy,
% Inc., under a cooperative agreement with the NSF on behalf of the
% Gemini partnership: the National Science Foundation (United States),
% National Research Council (Canada), CONICYT (Chile), Ministerio de
% Ciencia, Tecnolog\'{i}a e Innovaci\'{o}n Productiva (Argentina),
% Minist\'{e}rio da Ci\^{e}ncia, Tecnologia e Inova\c{c}\~{a}o (Brazil),
% and Korea Astronomy and Space Science Institute (Republic of Korea).
% %
% Observations in the paper made use of the High-Resolution Imaging
% instrument Zorro at Gemini-South. Zorro was funded by the NASA
% Exoplanet Exploration Program and built at the NASA Ames Research
% Center by Steve B. Howell, Nic Scott, Elliott P. Horch, and Emmett
% Quigley.
% %
% This research has made use of the VizieR catalogue access tool, CDS,
% Strasbourg, France. The original description of the VizieR service was
% published in A\&AS 143, 23.
% %
% This work has made use of data from the European Space Agency (ESA)
% mission {\it Gaia} (\url{https://www.cosmos.esa.int/gaia}), processed
% by the {\it Gaia} Data Processing and Analysis Consortium (DPAC,
% \url{https://www.cosmos.esa.int/web/gaia/dpac/consortium}). Funding
% for the DPAC has been provided by national institutions, in particular
% the institutions participating in the {\it Gaia} Multilateral
% Agreement.
%
% (Some of) The data presented herein were obtained at the W. M. Keck
% Observatory, which is operated as a scientific partnership among the
% California Institute of Technology, the University of California and
% the National Aeronautics and Space Administration. The Observatory was
% made possible by the generous financial support of the W. M. Keck
% Foundation.
% The authors wish to recognize and acknowledge the very significant
% cultural role and reverence that the summit of Maunakea has always had
% within the indigenous Hawaiian community.  We are most fortunate to
% have the opportunity to conduct observations from this mountain.
%
% \newline
%

\software{
  \texttt{astrobase} \citep{bhatti_astrobase_2018},
  %\texttt{astroplan} \citep{astroplan2018},
  \texttt{astropy} \citep{astropy_2018},
  \texttt{astroquery} \citep{astroquery_2018},
  %\texttt{BATMAN} \citep{kreidberg_batman_2015},
  \texttt{cdips-pipeline} \citep{bhatti_cdips-pipeline_2019}
  \texttt{corner} \citep{corner_2016},
  %\texttt{emcee} \citep{foreman-mackey_emcee_2013},
  \texttt{exoplanet} \citep{exoplanet:agol19}
  \texttt{exoplanet} \citep{exoplanet:exoplanet}, and its
  dependencies \citep{exoplanet:agol19, exoplanet:kipping13, exoplanet:luger18,
  	exoplanet:theano}.
  \texttt{IPython} \citep{perez_2007},
	\texttt{lightkurve} \citep{lightkurve_2018},
  \texttt{matplotlib} \citep{hunter_matplotlib_2007}, 
  \texttt{MESA} \citep{paxton_modules_2011,paxton_modules_2013,paxton_modules_2015}
  \texttt{numpy} \citep{walt_numpy_2011}, 
  \texttt{pandas} \citep{mckinney-proc-scipy-2010},
  \texttt{pyGAM} \citep{serven_pygam_2018_1476122},
  \texttt{PyMC3} \citep{salvatier_2016_PyMC3},
  %\texttt{radvel} \citep{fulton_radvel_2018},
  %\texttt{scikit-learn} \citep{scikit-learn},
  \texttt{scipy} \citep{jones_scipy_2001}.
  \texttt{tesscut} \citep{brasseur_astrocut_2019},
  \texttt{wotan} \citep{hippke_wotan_2019}.
}


\facilities{
 	{\it Astrometry}:
 	Gaia \citep{gaia_collaboration_gaia_2016,gaia_collaboration_gaia_2018}.
 	{\it Imaging}:
  Second Generation Digitized Sky Survey,
 	Keck:II~(NIRC2; \url{www2.keck.hawaii.edu/inst/nirc2}).
 	%Gemini:South~(Zorro; \citealt{scott_nessi_2018}.
 	{\it Spectroscopy}:
 	Keck:I~(HIRES; \citealt{vogt_hires_1994}).
% 	Euler1.2m~(CORALIE),
% 	ESO:3.6m~(HARPS; \citealt{mayor_setting_2003}).
 	{\it Photometry}:
% 	CTIO:1.0m (Y4KCam),
% 	Danish 1.54m Telescope,
% 	El Sauce:0.356m,
% 	Elizabeth 1.0m at SAAO,
% 	Euler1.2m (EulerCam),
% 	Magellan:Baade (MagIC),
% 	Max Planck:2.2m	(GROND; \citealt{greiner_grond7-channel_2008})
% 	NTT,
% 	SOAR (SOI),
 	TESS \citep{ricker_transiting_2015}.
% 	TRAPPIST \citep{jehin_trappist_2011},
% 	VLT:Antu (FORS2).
}

%
% The following are entries from Table 1 that are not otherwise cited
% in the text
%
% \nocite{wilson_wasp-4b_2008}
% \nocite{gillon_improved_2009}
% \nocite{winn_transit_2009}
% \nocite{hoyer_tramos_2013}
% \nocite{dragomir_terms_2011}
% \nocite{sanchis-ojeda_starspots_2011}
% \nocite{nikolov_wasp-4b_2012}
% \nocite{ranjan_atmospheric_2014}
% \nocite{huitson_gemini_2017}

% \input{WASP-4b_transit_time_table.tex}
% \input{WASP-4b_rv_table.tex}
% \input{model_fit_table.tex}
% \input{rv_model_posterior_table.tex}
% \input{pdot_table.tex}

\clearpage
\startlongtable
\begin{deluxetable*}{lrrrrrrrr}
%
%\tabletypesize{\scriptsize}
%
\tablenum{1}
%
\tablecaption{Model Comparison.}
\label{tab:modelcompare}
%
\tablehead{
\colhead{Description} &
\colhead{$N_{\rm s}$} &
\colhead{$N_{\rm \ell}$} &
\colhead{$N_{\rm data}$} &
\colhead{$N_{\rm param}$} &
\colhead{$\chi^2$} &
\colhead{$\chi_{\rm red}^2$} &
\colhead{BIC} &
\colhead{$\Delta$BIC}
}
% pasted from
% /Users/luke/Dropbox/proj/billy/results/PTFO_8-8695_results/20200428_v0/bic_table_data.tex
% 
% Burnham and Anderson 2004.
% "Models having i ≤ 2 have substantial support (evidence), those in which 4 ≤
% i ≤ 7 have considerably less support, and models having i > 10 have
% essentially no support"
\startdata
Favored    & 2 &  2 &   2585 &      17 &  3230.2 &     1.258 &  3363.7 &     0.0 \\
\hline
Weakly favored &  3 &  3 &   2585 &      21 &  3203.4 &     1.249 &  3368.4 &     4.7 \\
---               & 3 &  2 &   2585 &      19 &  3222.9 &     1.256 &  3372.2 &     8.4 \\
\hline
Disfavored      & 2 &  3 &   2585 &      19 &  3244.9 &     1.265 &  3394.2 &    30.4 \\
---              & 2 &  1 &   2585 &      15 &  3410.6 &     1.327 &  3528.5 &   164.7 \\
---             &  3 &  1 &   2585 &      17 &  3396.4 &     1.323 &  3530.0 &   166.3 \\
---             & 1 &  2 &   2585 &      15 &  4158.6 &     1.618 &  4276.4 &   912.7 \\
---             & 1 &  3 &   2585 &      17 &  4147.4 &     1.615 &  4281.0 &   917.2 \\
---            &  1 &  1 &   2585 &      13 &  4313.5 &     1.677 &  4415.6 &  1051.9 \\
\enddata
%
\tablecomments{
	$N_{\rm s}$ and $N_{\rm \ell}$ are the number of harmonics at the short and long periods, respectively.
	$N_{\rm data}$ is the number of fitted flux measurements.
	$N_{\rm param}$ is the number of free parameters in the model.
	The Bayesian information criterion (BIC) and the difference from the maximum $\Delta {\rm BIC}$ are also listed.
}
\vspace{-1cm}
\end{deluxetable*}

% Table of best fit parameters
\startlongtable
\begin{deluxetable*}{lllrrrrr}
%
\tablecaption{ Best-fit model priors and posteriors. }
\label{tab:posterior}
%
%\tabletypesize{\scriptsize}
%
\tablenum{2}
%
\tablehead{
  \colhead{Param.} & 
  \colhead{Unit} &
  \colhead{Prior} & 
  \colhead{Median} & 
  \colhead{Mean} & 
  \colhead{Std{.} Dev.} &
  \colhead{3\%} &
  \colhead{97\%}
}
% /Users/luke/Dropbox/proj/timmy/results/TOI_837_transit_phot_results/20200515/posterior_table_clean_transit.tex
\startdata
$P_{\rm s}$ & d & $\log\mathcal{N}(8.3247; 0.0050)$ & 8.3248321 & 8.3248295 & 0.0003356 & 8.3242288 & 8.3254817 \\
$t_{\rm s}^{(1)}$ & d & $\log\mathcal{N}(1574.273800; 0.0050)$ & 1574.2727299 & 1574.2727323 & 0.0010770 & 1574.2707084 & 1574.2747189 \\
$\log R_{\rm p}/R_\star$ & -- & $\mathcal{U}(0.01; 1.00)$ & -2.43072 & -2.236 & 0.55106 & -2.74890 & -0.99823 \\
$b$ & -- & $\mathcal{U}(0; 1+R_{\mathrm{p}}/R_\star)$ & 0.9729 & 1.0221 & 0.1533 & 0.9165 & 1.2972 \\
$u_1$ & -- & (2) & 0.335 & 0.335 & 0.088 & 0.187 & 0.469 \\
$u_2$ & -- & (2) & 0.229 & 0.227 & 0.086 & 0.093 & 0.374 \\
Mean & -- & $\mathcal{U}(0.99; 1.01)$ & 1.000028 & 1.000028 & 0.000006 & 1.000016 & 1.000039 \\
$R_\star$ & $R_\odot$ & $\mathcal{T}(1.05; 0.06)$ & 1.05 & 1.05 & 0.06 & 0.94 & 1.16 \\
$\log g$ & cgs & $\log\mathcal{N}(4.48; 0.09)$ & 4.46 & 4.46 & 0.08 & 4.32 & 4.61 \\
$R_{\rm p}/R_\star$ & -- & -- & 0.09 & 0.13 & 0.14 & 0.06 & 0.37 \\
$\rho_\star$ & g$\ $cm$^{-3}$ & -- & 1.42 & 1.44 & 0.28 & 0.97 & 1.97 \\
$R_{\rm p}$ & $R_{\mathrm{Jup}}$ & -- & 0.90 & 1.38 & 1.44 & 0.58 & 3.87 \\
$a/R_\star$ & -- & -- & 17.33 & 17.35 & 1.11 & 15.28 & 19.33 \\
$\cos i$ & -- & -- & 0.06 & 0.06 & 0.01 & 0.05 & 0.08 \\
$T_{14}$ & hr & -- & 1.79 & 1.79 & 0.05 & 1.68 & 1.87 \\
$T_{13}$ & hr & -- & 0.27 & 0.28 & 0.12 & 0.01 & NaN \\
\enddata
\tablenotetext{}{(1) To convert mean TESS mid-transit time from BTJD to ${\rm BJD}_{\rm TDB}$, add 2{,}457{,}000.\\
  (2) Quadratic limb-darkening prior from \citet{exoplanet:kipping13}, implemented by \citet{exoplanet:exoplanet}.
}
\vspace{0cm}
\end{deluxetable*}

\clearpage

\bibliographystyle{yahapj}                            
\bibliography{bibliography} 


\listofchanges

\end{document}
