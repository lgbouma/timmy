We report the discovery of TOI 837b and its validation as a transiting planet.  We characterize the system using data from the NASA TESS mission, the ESA Gaia mission, ground-based photometry from El Sauce and ASTEP400, and spectroscopy from CHIRON, FEROS, and Veloce.  We find that TOI 837 is a $T=9.9$ mag G0/F9 dwarf in the southern open cluster IC 2602.  The star and planet are therefore $35^{+11}_{-5}$ million years old.  Combining the transit photometry with a prior on the stellar parameters derived from the cluster color-magnitude diagram, we find that the planet has an orbital period of $8.3\,{\rm d}$ and is slightly smaller than Jupiter ($R_{\rm p} = 0.77^{+0.09}_{-0.07} \,R_{\rm Jup}$).  From radial velocity monitoring, we limit $M_{\rm p}\sin i$ to less than 1.20 $M_{\rm Jup}$ (3-$\sigma$).  The transits either graze or nearly graze the stellar limb.  Grazing transits are a cause for concern, as they are often indicative of astrophysical false positive scenarios.  Our follow-up data show that such scenarios are unlikely.  Our combined multi-color photometry, high-resolution imaging, and radial velocities rule out hierarchical eclipsing binary scenarios.  Background eclipsing binary scenarios, though limited by speckle imaging, remain a 0.2\% possibility.  TOI 837b is therefore a validated adolescent exoplanet.  The planetary nature of the system can be confirmed or refuted through observations of the stellar obliquity and the planetary mass.  Such observations may also improve our understanding of how the physical and orbital properties of exoplanets change in time.
